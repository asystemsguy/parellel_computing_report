% @Author: Harshavardhan.Kadiyala
% @Date:   2019-01-03 05:46:59
% @Last Modified by:   Harshavardhan.Kadiyala
% @Last Modified time: 2019-01-03 13:11:53

\section{Methodology}

\subsection{Setup}

\subsubsection{Tinkerpop}

Apache TinkerPop defines a basic interface for property graphs that defines a series of basic methods that can be used to interact with a graph: add and remove vertices (resp. edges), retrieve vertex (resp. edges) by identifier (ID), retrieve vertices (resp. edges) by attribute value, etc.. Working only with TinkerPop in an application allows to use a graph database in a total implementation-agnostic way, making it possible to easily switch from a graph database to another one without adaptation efforts. It uses gremlin graph traversal language to model and analyze the graphs. Both Neo4j and OrientDB allows use of gremlin to create, modify, and query the databases. In this project, we use python-gremlin to create queries for pattern matching and use the same code to test performance in both systems. Apache TinkerPop defines a particular graph model, the property graph, which is basically a directed, edge-labeled, attributed, multi-graph:

\textbf{G = (V, I, VID , D, J, EID , P, Q, R, S)} - (1)
where 
\begin{itemize}
\item V : a set of vertices, 
\item I : a set of vertices identifiers, 
\item VID : V -> I a function that associates each vertex to its identifier, 
\item D : a set of directed edges, 
\item J : a set of edges identifiers, 
\item EID : D -> J a function that associates each edge to its identifier, 
\item P (resp. R) is the vertices (resp. edges) attributes domain and 
\item Q (resp. S) the domain for allowed vertices (resp. edges) attributes values.
\end{itemize}


Apache TinkerPop provides a gremlin server which can be configured to use any gremlin-compliant graph databases using the database specific plugin. Clients can be in any programming language and makes contacts with the server to execute queries. We use python in this project to access the gremlin server.

Gremlin graph traversal language initally creates a traversal object that represents a graph. It then provides five types of steps to operate and transform  that traversal. 

\begin{itemize}
\item map - Map the traverser to some object of type E (Can be a vertices, edge, or set of attributes) for the next step to process
\item flatMap - Map the traverser to some objects of type E (Can be a vertices, edge, or set of attributes) for the next step to process
\item filter - Map the traverser to either true or false based on a predicate, where false will not pass the traverser to the next step.
\item sideEffect - perform some operation on the traverser (like grouping vertices based on attribute value) and pass it to the next step. 
\item branch - split the traverser to all the traversals indexed by the M token.
\end{itemize}

Clients query the graph by creating a traversal for a graph and by applying chain of steps on the traversal.

\subsubsection{Neo4j}

We used Neo4j community 3.2.13 in this project along with gremlin server of version 3.3.4.

\subsubsection{OrientDB}

We use OrientDB version xx in this project along with gremlin server of version YYY.

\subsection{Workload}


\subsubsection{Database}

We use IMDB database with edges xx and vertices xxx. The schema of the database includes. 

\subsubsection(Queries)

We compared the performance of Neo4j and OrientDB for matching three patterns

pattern 1

Pattern is about xxx

We implamented pattern by 

pattern 2

Pattern is about xxx

We implamented pattern by 

pattern 3

Pattern is about xxx

We implamented pattern by 


