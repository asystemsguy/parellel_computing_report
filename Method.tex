\section{Background}

\subsubsection{Apache TinkerPop}

Apache TinkerPop is a graph computing framework for both graph databases (OLTP) and graph analytic systems (OLAP). It defines a series of basic methods that can be used to interact with a graph: add and remove vertices (resp. edges), retrieve vertex (resp. edges) by an identifier (ID), retrieve vertices (resp. edges) by attribute value, etc. Working only with TinkerPop in an application allows using a graph database in a total implementation-agnostic way, making it possible to easily switch from a graph database to another one without adaptation efforts. 

Apache TinkerPop defines a particular graph model, the property graph, which is basically a directed, edge-labeled, attributed, multi-graph:

\textbf{G = (V, I, VID , D, J, EID , P, Q, R, S)} - (1)
where 
\begin{itemize}
\item V : a set of vertices, 
\item I : a set of vertices identifiers, 
\item VID : V -> I a function that associates each vertex to its identifier, 
\item D : a set of directed edges, 
\item J : a set of edges identifiers, 
\item EID : D -> J a function that associates each edge to its identifier, 
\item P (resp. R) is the vertices (resp. edges) attributes domain and 
\item Q (resp. S) the domain for allowed vertices (resp. edges) attributes values.
\end{itemize}

\subsubsection{Gremlin}

Apache TinkerPop uses the Gremlin graph traversal language to model and analyze the graphs. It also provides a gremlin server which can be configured to use any gremlin-compliant graph databases using the database specific plugin. Clients can be in any programming language and make contacts with the server to execute queries. We use python in this project to access the gremlin server.

Gremlin graph traversal language initially creates a traversal object that represents a graph. It then provides five types of steps to operate and transform that traversal. 

\begin{itemize}
\item map - Map the traverser to some object of type E (Can be a vertex, edge, or set of attributes) for the next step to the process
\item flatMap - Map the traverser to some objects of type E (Can be a vertex, edge, or set of attributes) for the next step to the process
\item filter - Map the traverser to either true or false based on a predicate, where false will not pass the traverser to the next step.
\item sideEffect - perform some operation on the traverser (like grouping vertices based on attribute value) and pass it to the next step. 
\item branch - split the traverser to all the traversals indexed by the M token.
\end{itemize}

Clients query the graph by creating a traversal for a graph and by applying chain of steps on the traversal.

\section{Goal of the study}

Main goal of this study is to empirically compare performance achieved by two popular open-source graph databases: Neo4j and OrientDB (according to ~\cite{dbranking}) for pattern-matching queries.

\section{Methodology}



\subsection{Workload}

\subsubsection{Database}

We use a large real-world IMDB database~\cite{IMDb96:online} with 29160968 edges and 5002524 vertices. The schema is consists of nodes: Movie, Genre,
Actress, Actor and Director. An edge is only possible between a Movie type vertex and a non-Movie type vertex. 
The schema of the database is shown in the figure~\ref{fig:schema}. 

\begin{figure}[t]
%\vspace{0.2in}
\centering
\resizebox{0.2\linewidth}{!}{
\includegraphics[width=0.8\linewidth]{Images/IMDB_schema}
}
\caption{IMDB database schema.}
\label{fig:schema}
\centering
\end{figure}

\subsubsection{Queries}

We compare the performance of below pattern matching queries when executed with Neo4j or OrientDB as a database. These Queries are implemented in gremlin-python using a multi-threaded approach with each thread searching a particular subset of the graph for the patterns. 

\begin{itemize}

\item Pattern 1

The first pattern is to find all actresses, actors, and directors that worked together on at least three different movies.

\item Pattern 2

The pattern is to find all actresses, actors, and directors that worked together on two movies produced in 2009 and 2011 respectively.

\item Pattern 3

The pattern is to find all the actresses, actors, and directors that worked together at least in two different movies that fall under a similar genre.

\end{itemize}
