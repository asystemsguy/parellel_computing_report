\section{Introduction}

-> Nosql 
-> why graph databases are important
-> Different options available
-> What we do in this report
-> Outline of whats next

Traditional SQL databases with their relational model have several disadvantages in covering current application domains. This has lead to the development of NoSQL databases. In the current market, there are many types of NoSQL databases. Each of them differs in their data model. Currently, popular data-models include wide-column stores such as BigTable; Document stores, which store semi-structured data(e.g MongoDB); key-value stores, which implement a persistent key to value hashtables; and Graph databases, which supports the representation of data as vertices and edges.

Graph data-structure is widely used in different applications related to chemistry, Biology, web mining, and semantic web. The efficent storing and processing of graphs became important. The graph databases which nativly stores the graphs can help these applications. Hence these graph databases have risin into prominance.

There are many graph databases that are available commertially and open-source. These graph databases implement various optimizations that can influance different charecteristics of graph processing. Two most popular opensource graph databases are neo4j and orientDb. In this report, we are going to compare these databases based on the pattern matching performance on a large real world graph. 

Neo4j is a graph database which natively stores the graph nodes on the disk. OrientDB supports other data-models such as key-value store, document store along with graph. 


Later in the report, in section 2, we will describe existing work that compares graph databases. In section 3, we will describe our methodology for evaluvation. In section 4, we will present our results.   




