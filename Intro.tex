\section{Introduction}

Traditional SQL databases with their relational model have several limitations in covering current application domains. This has lead to the development of NoSQL databases. In the current market, there are many types of NoSQL databases. Each of them differs in their data model. Currently, popular data-models include wide-column stores such as BigTable~\cite{chang2008bigtable}; Document stores, which store semi-structured data(e.g MongoDB~\cite{chodorow2013mongodb}); key-value stores, which implement a persistent key to value hashtables; and Graph databases, which supports the representation of data as vertices and edges.

A graph is one of the fundamental data abstractions in computer science. As such, there are many applications of graphs. Virtually every graph application has a need to store and query the graph. Recently there has been increased interest in graphs to represent social networks, biological networks, website link structures, and others. Within the field of biology itself, there are many uses for graphs, including metabolic networks, protein-protein interaction networks, chemical structure graphs, gene clusters, and genetic maps. Graphs truly are one of the most useful structures for modelling objects and interactions. Efficiently storing and manipulating graphs have become important as a large number of applications using graphs. Graph databases have risen into popularity to satisfy these requirements.

There are many graph databases that are available commercially and with an open-source licence. Each of them comes with its own characteristics and functionalities. For example, Titan~\cite{jouili2013empirical} and OrientDB~\cite{tesoriero2013getting} were developed to be easily distributed among multiple machines. Others natively provide their own query languages, such as Neo4j~\cite{holzschuher2013performance} with Cypher, or Sones 8 with GraphQL~\cite{vazquez2017improving}. Some databases are also compliant with a standard language, such as AllegroGraph9~\cite{abburu2013format} with SPARQL~\cite{abburu2013format}. In this multitude of solutions, some are characterized by a more generic data structure, like HypergraphDB10 or SonesDB that allow storing hypergraphs. 

Main goal of this study is to empirically compare performance achieved by two most popular open-source graph databases: Neo4j and OrientDB (according to ~\cite{dbranking}) for pattern-matching queries. To achieve that we decided to implement the pattern-matching queries in a manner that can  run on both the databases without any modification. This would be fair way to compare the databases and also this would be less tedious since we need to learn only one language instead of two as compared to first approach. Tinkerpop ~\cite{tinkerpop} framework enables one to write the queries in Gremlin language (designed according to the "write once, run anywhere" philosophy ~\cite{gremlin}). Since both Neo4j and OrientDB are TinkerPop compliant, we implement queries in Gremlin-Python that run on both databases. 


Later in the report, in section 2, we will describe existing work that compares graph databases. In section 3, we will describe our methodology for evaluation in detail. In section 4, we will present our results.   




