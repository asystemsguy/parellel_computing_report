\section{Introduction}

Battery-driven devices such as smart phones, tablets and wearables are now commonplace in our lives. Android has become one of the dominant mobile platforms used in these devices. Android app repositories, such as Google Play, have created a fundamental shift in the way software is delivered to consumers, with thousands of apps added and updated on a daily basis. Various mobile app markets offer a wide range of apps from entertainment, business, health care and social life. Mobile phones today are filled with large number of apps. On an average each person contains 35 apps in a smart phone~\cite{thinkwithgoogle}. 

These modern applications have sophisticated features similar to desktop applications and use power hungry functions such as GPS. From a user’s perspective, this produces tangible and pertinent problems. The use of energy-draining apps could quickly leave a user with empty battery, preventing from using the smart-phone even for phone calls. In addition, having and running such apps might require frequent battery re-charges. This represents a problem because modern battery’s life is quite limited, often to a finite amount of charging cycles (for Lithium-ion batteries), ranging between 300 and 500 cycles (with only 100-200 cycles after a mid-life point) and gradually decreasing with time~\cite{linares2014mining}.

Though many apps are installed in a phone, research on usage patterns of mobile applications in general media~\cite{techcrunch85}~\cite{techcrunch63} suggests that users use only few apps frequently. These important set of applications have high priority from user's perspective.
But remaining apps installed in the phone also consume energy for background process such as ads~\cite{stevens2012investigating}. But in today's android system, there is no feature to give priorities for certain apps and restrict power consumption by other less important apps. 

It is often the case that many apps provide similar features, but with different implementation choices, thereby impacting their energy consumption. For instance, Google Play hosts dozens of highly rated weather apps, providing almost identical features. These apps share highly similar features and ratings, but do not provide any readily available information as to their energy costs to help the user make an informed decision about which apps to use. This is why user centric energy management is required to assign fixed energy budgets for mobile applications, which will force these applications to use right design and implementation choices to save energy for the user.



% - energy - usage patterns - lot of energy is used by low priority - leaving no energy for high priority - less user control over the Apps in terms of energy.

\section{Problem statement}

Mandatory access control for energy(EnergyMAC) usage based on user preferences is a required but missing feature in energy constrained android mobile devices. Our project is aimed at providing user control over energy consumption of applications by using fixed energy credits. 

 In our system, users will allocate fixed energy credits to applications according to there relative importance. When applications use energy hungry system features such as GPS or network, our energy accounting system will check for availability of required energy credits to perform that operation and reject the permission to access system resource if enough credits are not available. This will force applications to adapt their functionalities for assigned energy credits in-order stay in a working state for a long time. 

%problem statement
%-----------------
%To give more control - implement mac for android apps - without considerable effects on performance

\section{Related work}
%Energy mesurement
%Energy Accounting

There is a large body of work on energy consumption of Android apps. Prior related studies can be categorized in two ways: power modeling and power measurement. Research in power modeling suggests estimating the energy usage of mobile devices or apps in the absence of hardware power
monitors ~\cite{hao2013estimating}, ~\cite{li2013calculating}. These software-based approaches build models and capture model parameters from programs using static-analysis techniques. EnergyMac on other hand uses these models to enforce strict energy access limits on applications.

Studies in power measurement make use of specialized hardware, such as Monsoon~\cite{yoon2012appscope}, and map the sampled measurements to execution traces to determine an app's energy consumption at various granularities. To the best of our knowledge, EnergyMac is the first work that has attempted to implement a access control for Android apps according to their energy consumption. 



%------------
%works which look at measuring energy, assigning energy consumption on each other, controlling energy consumption - no body looks at mac every one looks at blaming they are complimentary for our approach.
